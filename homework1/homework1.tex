\documentclass[a4paper,12pt]{article}
\usepackage{amssymb}
\usepackage{amsmath}
\usepackage{enumerate}
\usepackage{geometry}
\usepackage{ dsfont }

\geometry{margin=1in}
\begin{document}

\section*{Homework 1}

\begin{enumerate}
    \item In each of the following situations, indicate wherether $f = O(g)$, or $f = \Omega(g)$, or both (in which case $f = \Theta(g)$)

    \begin{tabular}{lll|r}
      ~ & $f(n)$ & $g(n)$ & $O$, $\Omega$, or $\Theta$?\\
      (a) & $n - 100$ & $n - 200$ & $f = \Theta(g)$ \\
      (b) & $n^{1/2}$ & $n^{2/3}$ & $f = O(g)$\\
      (c) & $100n + \log n$ & $n + (\log n)^2$& $f = \Theta(g)$ \\
      (d) & $n \log n $ & $10n \log 10n$ & $f = \Theta(g)$\\
      (e) & $\log 2n$ & $\log 3n$ & $f = \Theta(g)$ \\
      (f) & $10 \log n$ & $\log(n^2)$ & $f = \Theta(g)$ \\
      (g) & $n^{1.01}$ & $n \log^2 n$ & $f = \Omega(g)$\\
      (h) & $n^2 / \log n$ & $n(\log n)^2$ & $f = \Omega(g)$\\
      (i) & $n^{0.1}$ & $(\log n)^{10}$ & $f = \Omega(g)$\\
      (j) & $(\log n)^{\log n}$ & $n / \log n$ & $f = \Omega(g)$\\
      (k) & $\sqrt(n)$ & $(\log n)^3$ & $f = O(g)$\\
      (l) & $n^{1/2}$ & $5^{\log_2 n}$ & $f = O(g)$\\
      (m) & $n2^n$ & $3n$ & $f = \Omega(g)$\\
      (n) & $2^n$ & $2^{n+1}$ & $f = \Theta(g)$ \\
      (o) & $n!$ & $2^n$ & $f = \Omega(g)$ \\
      (p) & $(\log n)^{\log n}$ & $2^{(\log_2 n)^2}$ & $f = O(g)$ \\
      (q) & $\sum_{i = 1}^n i^k$ & $n^{k+1}$ & $f = O(g)$\\
    \end{tabular}
    \item Dasgupta Problem 1.13

    Is the difference of $5^{30,000}$ and $5^{30,000}$ a multiple of $31$?
    \\\\
    We know that $5^3 = 125$ and $31*4 = 124$ so it follows that.
    $$5^{30,000} \equiv (5^3)^{10,000} \equiv (125)^{10,000} \equiv 1^{10,000} \equiv 1 (\text{mod } 31)$$
    We also know that $6^6 = 46,656$, $31\times 1,505 = 46,655$, and $123,456/6=20,576$. So it follows that
    $$6^{123,456} \equiv (6^6)^{20,576} \equiv (46,655)^{20,576} \equiv 1^{20,576} \equiv 1 (\text{mod } 31)$$

    So it follows that

    $$5^{30,000} - 6^{123,456} \equiv 1 - 1 \equiv 0 (\text{mod } 31)$$

    So the difference is divisible by 31.

    \item Levitin Problem 2.1.5b

    Prove the alternative formula for the number of bits in the binary representation of a positive integer $n$:

    $$b = \lceil \log_2(n+1)\rceil$$

    We observe that 1) a number $2^{b-1} \,|\, b \in \mathds{Z}$ is representable in binary with $b$ bits. We also observe that 2) any number $k$ such that $2^{b-1} \leq k < 2^b$ is also expressible with $b$ bits. Given a number $n$, it follows that $n = 2^{b-1}\,|\, b \in \mathds{Z}$ or $n \not = 2^{b-1}\,|\, b \in \mathds{Z}$. Given the first case, it follows that $n+1 < 2^b$ unless $n = 0$ or $n = 1$. If $n = 0$ or $n = 1$, then $\lceil \log_2(n+1)\rceil = 1$, which is the number of bits required to express 0 or 1. Otherwise, since $n + 1 < 2^b$, by observation 2), we see that $n + 1$ is also expressible with $b$ bits. We also know that since $n+1 < 2^b$ that $b -1 < \log_2 (n+1) < b$. Since $b \in \mathds{Z}$, $\lceil \log_2(n+1)\rceil = b$.

    Given the second case where $n \not = 2^{b-1}\,|\, b \in \mathds{Z}$, 
\end{enumerate}




\end{document}